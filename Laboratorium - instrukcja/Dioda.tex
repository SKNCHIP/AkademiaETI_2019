\documentclass{article}
\usepackage[MeX]{polski}
\usepackage[utf8]{inputenc}
\usepackage{graphicx}
\usepackage{adjustbox}
\linespread{1.3}
\title{Dzielnik częstotliwości}
\begin{document}
\maketitle
\section{Migająca dioda}
W tym zadaniu poruszony zostanie temat dzielnika częstotliwości. Do obserwacji jego działania zostanie użyta zwykła dioda. Częstotliwość generowana przez urządzenie to 50MHz, co oznacza, że dioda powinna migać co około $2*10^-8$s, co jest niemożliwe do zauważenia dla człowieka. Jednak poprzez zastosowanie odpowiedniego układu jakim jest dzielnik częstotliwości, można zwiększyć ten czas do np. 1s.
Dzielnik częstotliwości jest układem służącym do zmniejszenia częstotliwości. Układ ten, dzieli częstotliwości zegara wejściowego, czyli częstotliwość generowaną przez urządzeniu i na wyjściu uzyskujemy odpowiednio podzieloną wartość. W układzie przedstawionym na zajęciach oprócz wejścia i wyjścia występuje przycisk reset. Działa on asynchronicznie, czyli wartość na wyjściu ukłądu zostanie wyzerowana w momencie naciśnięcia przycisku, niezależnie od zegara.
\end{document}