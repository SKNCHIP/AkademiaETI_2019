\documentclass{article}
\usepackage[MeX]{polski}
\usepackage[utf8]{inputenc}
\usepackage{graphicx}
\usepackage{adjustbox}
\linespread{1.3}
\title{Wyswietlacz 7 segmentowy}
\begin{document}
\maketitle
\section{Zasada działania}
Kolejnym układem którego działanie zostanie zaprezentowane jest wyświetlacz 7 segmentowy. Taki wyświetlacz, dzięki swojej prostocie, używany jest głównie do wyświetlania liczb. Pojedynczy wyświetlacz składa się z 7 segmentów, które zostają podświetlane zależnie od danych jakie chcemy wyświetlić..(obrazek). W laboratorium występuje  poczwórny wyświetlacz. Taki wyświetlacz posiada jedną 8 bitową linię danych oraz 4 linie aktywujący pojedynczy wyświetlacz. Z tego powodu jeśli na każdym wyświetlaczu chcemy wyświetlić inny znak(np. zrobić stoper) musimy zastosować pewną sztuczkę. Mianowicie dane na każdy z wyświetlaczy należy wysyłać kolejno. Najpierw wysyłamy na pierwszy wyświetlacz, następnie na drugi itp. Aby znak został wyświetlony tylko na jednym wyświetlaczu, na linii aktywującej konkretny wyświetlacz powinna wystąpić wartość 0, a na innych liniach 1. Aby taki układ działał poprawnie, musi działać z pewną częstotliwością. Częstotliwość ta musi być odpowiednio wysoka, aby ludzkie oko nie wychwyciło migotania obrazu. W tym zadaniu została przyjęta 1kHz. Do całości tego zadania został użyty dzielnik częstotliwości z poprzedniego zadania, aby częstotliwość taktowania wyświetlacza nie była zbyt wysoka. 
\end{document}